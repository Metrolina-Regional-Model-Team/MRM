% Options for packages loaded elsewhere
\PassOptionsToPackage{unicode}{hyperref}
\PassOptionsToPackage{hyphens}{url}
%
\documentclass[
]{article}
\usepackage{amsmath,amssymb}
\usepackage{iftex}
\ifPDFTeX
  \usepackage[T1]{fontenc}
  \usepackage[utf8]{inputenc}
  \usepackage{textcomp} % provide euro and other symbols
\else % if luatex or xetex
  \usepackage{unicode-math} % this also loads fontspec
  \defaultfontfeatures{Scale=MatchLowercase}
  \defaultfontfeatures[\rmfamily]{Ligatures=TeX,Scale=1}
\fi
\usepackage{lmodern}
\ifPDFTeX\else
  % xetex/luatex font selection
\fi
% Use upquote if available, for straight quotes in verbatim environments
\IfFileExists{upquote.sty}{\usepackage{upquote}}{}
\IfFileExists{microtype.sty}{% use microtype if available
  \usepackage[]{microtype}
  \UseMicrotypeSet[protrusion]{basicmath} % disable protrusion for tt fonts
}{}
\makeatletter
\@ifundefined{KOMAClassName}{% if non-KOMA class
  \IfFileExists{parskip.sty}{%
    \usepackage{parskip}
  }{% else
    \setlength{\parindent}{0pt}
    \setlength{\parskip}{6pt plus 2pt minus 1pt}}
}{% if KOMA class
  \KOMAoptions{parskip=half}}
\makeatother
\usepackage{xcolor}
\usepackage[margin=1in]{geometry}
\usepackage{graphicx}
\makeatletter
\def\maxwidth{\ifdim\Gin@nat@width>\linewidth\linewidth\else\Gin@nat@width\fi}
\def\maxheight{\ifdim\Gin@nat@height>\textheight\textheight\else\Gin@nat@height\fi}
\makeatother
% Scale images if necessary, so that they will not overflow the page
% margins by default, and it is still possible to overwrite the defaults
% using explicit options in \includegraphics[width, height, ...]{}
\setkeys{Gin}{width=\maxwidth,height=\maxheight,keepaspectratio}
% Set default figure placement to htbp
\makeatletter
\def\fps@figure{htbp}
\makeatother
\setlength{\emergencystretch}{3em} % prevent overfull lines
\providecommand{\tightlist}{%
  \setlength{\itemsep}{0pt}\setlength{\parskip}{0pt}}
\setcounter{secnumdepth}{-\maxdimen} % remove section numbering
\ifLuaTeX
  \usepackage{selnolig}  % disable illegal ligatures
\fi
\usepackage{bookmark}
\IfFileExists{xurl.sty}{\usepackage{xurl}}{} % add URL line breaks if available
\urlstyle{same}
\hypersetup{
  pdftitle={Feedback},
  hidelinks,
  pdfcreator={LaTeX via pandoc}}

\title{Feedback}
\author{}
\date{\vspace{-2.5em}}

\begin{document}
\maketitle

{
\setcounter{tocdepth}{4}
\tableofcontents
}
Trip tables created in the Destination Choice step depend, to some
degree, on the travel time between the zones. Feedback loops refers to
the process of taking the congested travel time from the traffic
assignment step and feeding it back to the trip distribution process.
These travel times are iteratively fed back to the trip distribution
step to recalculate the trip tables. The purpose of the feedback loop is
to include more ``realistic'' travel times in the distribution, mode
choice, and highway assignment steps.

\subsection{Feedback in Destination
Choice}\label{feedback-in-destination-choice}

\section{Calculation of Feedback Travel
Times}\label{calculation-of-feedback-travel-times}

Feedback travel times are calculated for the highway and transit models
to more accurately represent the effect of congestion on the
transportation system.

\subsection{Highway travel times and
skims}\label{highway-travel-times-and-skims}

Following highway assignment, new peak highway skims are built using an
average of travel times from the assignment and from the previous run.
Due to the nature of regional modeling, highway delay can be estimated
accurately on an aggregate basis, but can sometimes produce unreasonable
results on a link-by-link basis. Since link delay in the model is
impacted by the length of the link, queuing onto upstream links (like
traffic on a congested ramp spilling back onto a freeway) cannot be
accounted for. Highly congested links may also produce unrealistically
low speeds (under 1 mph) that cannot occur. To account for this, a
minimum speed of 10\% of free-flow speed is applied to each link during
feedback.

After the minimum speed is applied, the travel times are averaged with
travel times from the previous iteration. An average of the two
iterations is used to prevent large oscillations in the travel time, but
slows convergence. These travel times are used to create a new set of
morning peak highway travel times and impedances, which is used for the
next feedback loop. A total of three iterations are used to re-run peak
distribution, peak mode choice, and peak assignment. On the third and
final pass, the evening peak and off-peak assignments are completed as
well.

\subsection{Transit travel times and
skims}\label{transit-travel-times-and-skims}

Initial transit travel times are determined using a look-up table of
average bus speeds by facility type and area type. A minimum speed of
90\% of the background traffic speed is also calculated and used if it
is faster than the lookup speed. Buses typically operate on the same
facilities as all other traffic and are impacted by congestion just as
much as all other vehicles (except for exclusive guideways, which are
unaffected by the feedback process). Therefore, these calculations are
repeated and subsequently updated for each feedback iteration. In cases
where background traffic is slower in the second iteration than the
first, buses should experience a similar effect. Conversely, if
background traffic travels faster due to less congestion, buses should
benefit as well. This will only occur, however, if the minimum speed
(90\% of background traffic speed) is lower than the transit lookup
speed, since it uses the higher of the two.

\end{document}
